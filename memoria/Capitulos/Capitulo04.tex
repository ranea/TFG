%**************************
\chapter{Conclusiones y vías futuras}
\label{ch:Conclusiones y vías futuras}
%**************************

% Las conclusiones deberán incluir todas aquellas de tipo profesional y académico. Además, se deberá indicar si los objetivos han sido alcanzados totalmente, parcialmente o no alcanzados.
%
% Si hubiese posibles vías claras de desarrollo posterior sería interesante destacarlas aquí, poniéndolas en valor en el contexto inicial del trabajo.
%
% Finalmente vienen las conclusiones y recomendaciones, en las que, a partir de lo planteado en las secciones anteriores, se presentan los resultados del trabajo, según las hipótesis de partida que se exponían en la introducción. Según los casos, aquí se pueden incluir recomendaciones prácticas y futuras líneas de investigación.

Como hemos visto, la teoría de curvas elípticas es una teoría rica, sofisticada y extensa. Sin embargo, uno de sus aspectos más sorprendentes es su aplicación en la criptografía y como supone una mejora de los sistemas criptográficos ampliamente utilizados. Esta aplicación es un muy buen ejemplo de campo interdisciplinar entre las matemáticas y las ciencias de la computación.

Los objetivos propuestos inicialmente se han alcanzando en su totalidad, sin embargo, la criptografía con curvas elípticas abarca mucho más de lo tratado en este trabajo. Existen así numerosas vías de desarrollo posterior, en la que destacamos algunas:
\begin{itemize}
    \item La \emph{criptografía basada en identidad}, cuyos esquemas criptográficos más utilizados se basan en los emparejamientos bilineales de curvas elípticas, como el emparejamiento Weil estudiado en la sección~\ref{subs:Emparejamiento Weil}.
    \item Un estudio más profundo acerca de los ataques sobre el problema del logaritmo discreto con curvas elípticas estudiado en el apartado~\ref{sub:Problema del logaritmo discreto}. Como ya explicamos, la seguridad de los sistemas criptográficos viene determinada por los ataques conocidos y en este trabajo se realizó una mera introducción sobre ellos. Estos ataques presenten ideas muy interesantes y usan resultados sutiles para conseguir la mayor eficiencia.
    \item Las curvas hiperelípticas, una generalización de las curvas elípticas, y su aplicación en la criptografía. Como ya comentamos en la introducción, aún no se han conseguido sistemas basados en curvas hiperelípticas bien más seguros o bien más eficientes que los basados en curvas elípticas. Aún así, es un campo muy reciente y con gran camino por recorrer.
    \item Desarrollo de algoritmos más eficientes. Mucha de la investigación actual sobre criptografía con curvas elípticas se basa en la búsqueda de algoritmos más eficientes para mejorar la velocidad de estos sistemas. Así, se podría hacer un estudio del estado del arte y desarrollar un nuevo algoritmo que suponga un avance en este ámbito.
\end{itemize}
