%**************************
\chapter{Objetivos del trabajo}
\label{ch:Objetivos del trabajo}
%**************************

% En este apartado deberán aparecer con claridad los objetivos inicialmente previstos en la propuesta de TFG y los finalmente alcanzados con indicación de dificultades, cambios y mejoras respecto a la propuesta inicial. Si procede, es conveniente apuntar de manera precisa las interdependencias entre los distintos objetivos y conectarlos con los diferentes apartados de la memoria.

% Se pueden destacar aquí los aspectos formativos previos más utilizados.

Los objetivos inicialmente previstos eran:
\begin{itemize}
    \item Realizar un estudio en profundidad de las curvas elípticas sobre cuerpos finitos y la aritmética de sus puntos.
    \item Implementar algoritmos para trabajar con la aritmética del grupo de una curva elíptica.
    \item Implementar protocolos criptográficos basados en curvas elípticas.
\end{itemize}

Para alcanzar el primer objetivo, estudiamos primero el caso general, esto es, curvas elípticas sobre un cuerpo arbitrario, y depués particularizamos a cuerpos finitos. Como la teoría de curvas elípticas es muy extensa, no hemos podido abarcar todo lo que nos hubiera gustado y hemos tenido que elegir la parte más representativa de esta materia en relación con su aplicación en la criptografía.

Para alcanzar los dos objetivos relacionados con la implementación, previamente realizamos un estudio sobre la criptografía asimétrica con curvas elípticas, haciendo énfasis en los algoritmos y procedimientos que contiene y sus protocolos criptográficos. Análogamente al caso anterior, tuvimos que seleccionar los aspectos más importantes de la criptografía asimétrica con curvas elípticas debido a la multitud de detalles técnicos presentes en este campo.

Con este conocimiento, desarrollamos un programa informático capaz de operar con el grupo de puntos de una curva elíptica y trabajar con protocolos criptográficos.
Para ello también tuvimos que implementar la aritmética modular entre enteros y polinomios y la aritmética de cuerpos finitos. Los protocolos criptográficos implementados y utilizados en aplicaciones reales han tenido en cuenta una multitud de detalles técnicos relacionados con la eficiencia y la seguridad, de los cuales solo los más cruciales han sido tenido en cuenta en nuestra implementación particular.

En la tabla~\ref{tab:Objetivos alcanzandos} mostramos los objetivos finalmente alcanzados y el apartado de la memoria donde se trata cada objetivo.

\begin{table}[p]
  \myfloatalign
  \begin{tabularx}{\textwidth}{Xl} \toprule
    \tableheadline{Objetivo alcanzado} & \tableheadline{Localización}  \\
    \midrule
    Realizar un estudio en profundidad de las curvas elípticas y la aritmética de sus puntos y particularizarlo a curvas elípticas sobre cuerpos finitos. & Capitulo~\ref{ch:Desarrollo matemático} \\
    Realizar un estudio sobre la criptografía asimétrica con curvas elípticas, incluyendo algunos de sus algoritmos y protocolos más importantes. & Sección~\ref{sec:Criptografía con curvas elípticas} \\
    Implementar algoritmos para trabajar con la aritmética modular de enteros y polinomios, de cuerpos finitos y del grupo de una curva elíptica. & Sección~\ref{sec:Criptografía con curvas elípticas con Python} \\
    Implementar protocolos criptográficos basados en curvas elípticas. & Sección~\ref{sec:Criptografía con curvas elípticas con Python} \\
    \bottomrule
  \end{tabularx}
  \caption{Objetivos alcanzandos.}\label{tab:Objetivos alcanzandos}
\end{table}

Por otro lado, la tabla~\ref{tab:Asignaturas más relevantes} muestra las materias del Doble Grado en Matemáticas e Ingeniería Informática más relacionadas con el trabajo.

\begin{table}[p]
  \myfloatalign
  \begin{tabularx}{\textwidth}{X} \toprule
    %\tableheadline{Asignaturas} \\
    %\midrule
    % Matemáticas
    Álgebra I, II, III. \\
    Geometría I, II. \\
    Curvas y superficies. \\
    Álgebra conmutativa computacional. \\
    Teoría de números y criptografía. \\

    % Informática
    Fundamentos de programación. \\
    Metodología de la programación. \\
    Estructura de datos. \\
    Algorítmica. \\
    Fundamentos de ingeniería del software. \\
    Modelos avanzados de computación. \\
    Programación y diseño orientado a objetos. \\
    Seguridad y protección de sistemas informáticos. \\
    \bottomrule
  \end{tabularx}
  \caption{Materias del doble grado relacionadas con el trabajo.}\label{tab:Asignaturas más relevantes}
\end{table}
