%**************************
\chapter{Introducción}
\label{ch:Introducción}
%**************************

\subsection{Contextualización}
\label{sub:Contextualización}

% Contextualizar el trabajo explicando antecedentes importantes para el desarrollo realizado y efectuando, en su caso, un estudio de los progresos recientes.

Las curvas elípticas han sido estudiadas por los matemáticos desde la antigüedad y se han utilizado para resolver un abanico variado de problemas. Un ejemplo es el problema de los números congruentes que pregunta por la clasificación de los enteros positivos que aparecen como el aréa de un triangulo rectángulo cuyos lados sean números racionales. Otro ejemplo ha sido el Último Teorema de Fermat que afirma que la ecuación $x^n + y^n = z^n$ no tiene soluciones no triviales para $x, y$ y $z$ cuando $n$ es más grande que 2.

% TODO: a finales de la decada?
La aplicación de las curvas elípticas en la criptografía vino mucho después. En 1985, Neal Koblitz y Victor Miller propusieron independientemente utilizar las curvas elípticas para diseñar sistemas criptográficos de llave pública. Desde entonces, se han publicando una gran cantidad de estudios sobre la seguridad y la eficiencia de las implementaciones de la criptografía con curvas elípticas. A finales de la década de los 90, los sistemas con curvas elípticas empezaron a recibir aceptación comercial cuando organizaciones de estándares acreditadas especificaron protocolos con curvas elípticas y compañías privadas incluyeron dichos protocolos en sus productos de seguridad.

% TODO: debería citaar
Recientemente, se ha comenzado a investigar el uso de curvas hiperelípticas, una generalización de las curvas elíptipcas, para aplicarlas a sistemas criptográficos de llave pública. Sin embargo, estos sistemas son menos seguros o son menos eficientes que los sistemas que usan curvas elípticas.

Uno de los principales problemas al que se enfrenta la criptografía con curvas elípticas, y los principales criptosistemas de llave pública, son los ordenadores cuánticos. En 1994, Shor presentó un algoritmo para un ordenador cuántico capaz de calcular logaritmos discretos y factorizar enteros, principales problemas matemáticos en los que se basa la criptografía de llave pública, de forma eficiente. Sin embargo, hasta la fecha no se ha podido construir un ordenador cuántico con la capacidad suficiente para resolver instancias de estos problemas no triviales.

\subsection{Problema a abordar}
\label{sub:Problema a abordar}

% Describir el problema abordado, de forma que el lector tenga desde este momento una idea clara de la cuestión a resolver o del producto a desarrollar y una visión general de la solución alcanzada.

El problema abordado ha sido la realización de un estudio teórico-práctico sobre las curvas elípticas en criptografía y la implementación de diversos protocolos criptográficos con curvas elípticas.

En el estudio teórico, se abarca la teoría de curvas elípticas desde un punto de vista formal y riguroso. En primer lugar, se tratan las curvas elípticas sobre un cuerpo arbitrario y posteriormente se particulariza a curvas elípticas sobre cuerpos finitos.

%  ya que la literatura sobre este campo es muy extensa.
Nótese que un estudio exhaustivo sobre la teoría de curvas elípticas hubiera sido inviable debido a la enorme extensión que cubre este campo. Serge Lang, autor de esta materia, escribió en la introducción de uno de sus libros: <<It is possible to write endlessly on elliptic curves>>. Por ello, hemos elegido la parte más representativa de esta materia en relación con su aplicación en la criptografía.

En el estudio práctico, se introduce la criptografía de llave pública con curvas elípticas y se hace énfasis en cuestiones de seguridad y de implementación además de explicar algunos ejemplos de esquemas criptográficos con curvas elípticas. Análogamente al estudio teórico, la criptografía con curvas elípticas es un campo muy amplio y de la multitud de aspectos técnicos presentes hemos abarcado los más cruciales.

% TODO: revisar frase (sobre todo el así..)
Por ultimo, se ha desarrollado un programa informático que implementa diversos protocolos criptográficos con curvas elípticas y así aplicar el estudio teórico-práctico realizado. Este programa también permite hacer cálculos con curvas elípticas y cuerpos finitos y ha sido ampliamente testado y bien documentado.

\subsection{Técnicas matemáticas e informáticas}
\label{sub:Técnicas matemáticas e informáticas}

%  Exponer con claridad las técnicas y áreas matemáticas, así como los conceptos y herramientas de la ingeniería informática que se han empleado.

Las principales áreas matemáticas junto con los conceptos y resultados utilizados se han recogido en la tabla~\ref{tab:Herramientas matemáticas utilizadas}. Esta tabla no pretende ser exhaustiva, pero si mostrar una idea de los conocimientos que han servido de base para estudiar las curvas elípticas. A partir de estos conocimientos, se han definido conceptos, resultados y técnicas propias de la teoría de curvas elípticas no presentes en esta tabla.

\begin{table}[!h]
  \myfloatalign
  \begin{tabularx}{\textwidth}{lX} \toprule
    \tableheadline{Áreas matemáticas} & \tableheadline{Conceptos, resultados o técnicas}  \\
    \midrule
    Teoría de números & Los anillos de enteros módulo un entero positivo, el algoritmo extendido de Euclides y el teorema chino del resto. \\
    Teoría de cuerpos & Conceptos como la característica o la clausura algebraica de un cuerpo y los cuerpos finitos y el automorfismo de Frobenius. \\
    Teoría de grupos & Conceptos como el orden, la suma directa, el logaritmo discreto, los homomorfismos entre grupos o el núcleo de un homomorfismo y teoremas como el primer teorema de isomorfía, el teorema de lagrange para el orden de un grupo o el teorema de estructura para grupos abelianos finitos y los grupo de las raíces de $n$-ésimas de la unidad unidad y los grupos cíclicos finitos.\\
    Geometría algebraica & El concepto de curva, el espacio afín y el espacio proyectivo. \\
    \bottomrule
  \end{tabularx}
  \caption{Principales herramientas matemáticas utilizadas}  \label{tab:Herramientas matemáticas utilizadas.}
\end{table}

Por otro lado, los conceptos y las herramientas de la informáticas empleadas se han recogido en la tabla~\ref{Herramientas informáticas utilizadas}.

\begin{table}[!h]
  \myfloatalign
  \begin{tabularx}{\textwidth}{lX} \toprule
    \tableheadline{Áreas informáticas} & \tableheadline{Conceptos, resultados o técnicas}  \\
    \midrule
    Criptografía & Conceptos como criptosistema y su clasificación (de llave privada, pública o híbrido), firma digital o función hash. \\
    Teoría de la computación & Complejidad temporal y su clasificación en clases, notación $O$ para el análisis de algoritmos y nociones como la intractabilidad o el caso promedio. \\
    Programación & Diseño orientado a objetos (conceptos como clase, herencia o clases abstractas), desarrollo de pruebas o documentación. \\
    \bottomrule
  \end{tabularx}
  \caption{Principales herramientas informáticas utilizadas}  \label{tab:Herramientas informáticas utilizadas.}
\end{table}

% TODO: poner que estos conceptos aparecen en las materias del grado

% \subsection{Contenido de la memoria}
% \label{sub:Contenido de la memoria}
%
% % Sintetizar el contenido de la memoria.
%
% \begin{itemize}
%     \item Estudio EC
%     \item Estudio ECC
%     \item Implementacion EC y ECC
%     \item Aplicacion sobre las páginas de la ugr
% \end{itemize}

% \subsection{Principales fuentes}
% \label{sub:Principales fuentes}
%
% % Citar las principales fuentes consultadas.
%
% Lawrence.
% Waterloo.
% Menezes.
% Las referencias web.
