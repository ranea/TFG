%**************************
\chapter{Desarrollo informático}
\label{ch:Desarrollo informático}
%**************************

% TODO: completar introducción
En este capítulo haremos el desarrollo informático sobre criptografía con curvas elípticas. En el apartado~\ref{} hablaremos sobre protocolos criptográficos y explicaremos la implementación del programa desarrollado en el apartado~\ref{}.

% TODO: añadir ref si se utilizán más
Las referencias utilizadas para el desarrollo informático han sido sido~\cite{Hankerson:2003} y~\cite{Washington:2008}.

% TODO: ver si hacer una introducción de criptografía.

\section{Protocolos criptográficos}
\label{sec:Protocolos criptográficos}

% TODO: intro

\subsection{Problema del logaritmo discreto}
\label{sub:Problema del logaritmo discreto}

La dificultad del problema del logaritmo discreto es esencial para la seguridad de los esquemas criptográficos sobre curvas elípticas.

\begin{definicion}
    El \emph{problema del logaritmo discreto sobre curvas elípticas (ECDLP)} es: dado una curva elíptica $E$ definida sobre un cuerpo finito $\Fq$, un punto $P \in E(\Fq)$ de orden $n$ y un punto $Q \in <P>$, encontrar el entero $k \in [0, n - 1]$ tal que $Q = k P$. El entero $k$ se llama el \emph{logaritmo discreto de $Q$ respecto a la base $P$} y se denota $k = \log_p Q$.
\end{definicion}

Los parámetros de una curva elíptica para los esquema criptográficos deben ser elegidos con cuidado para resistir todos los ataques conocidos sobre el ECDLP.

El ataque más simple es el ataque por \emph{fuerza bruta}: probar todos los valores posibles de $k$ hasta dar con el válido. El tiempo de ejecucción es aproximadamente $n$ pasos en el peor caso y $n / 2$ pasos en el caso medio. Así, el ataque por fuerza bruta puede ser evitado tomando curvas elípticas con $n$ suficientemente largo (p. ej. $n > 2^{80}$).

El mejor ataque de propósito general conocido sobre el ECDLP es la combinación del \emph{algoritmo de Pohlig-Hellman} y del \emph{algoritmo rho de Pollard}, que tiene un complejidad temporal exponencial de $O(\sqrt{p})$ donde $p$ es el divisor primo más grande de $n$. Para resistir este ataque, se debe elegir curvas elípticas tal que $n$ sea divisible por un primo $p$ suficiente grande (p. ej $p > 2^{160}$).

Además se deben evitar ciertos tipos de curvas para los cuales existen ataques específicos más rápidos que el algoritmo rho de Pollard. Así pues, se deben evitar las curvas \emph{anómalas} (cuyo orden es de la forma $|E(\Fp)| = p$), curvas supersingulares (véase ~\ref{def:supersingular}), curvas con un orden de la forma $|E(\Fq)| = q - 1$ y curvas sobre $\Fm$ si $m$ es compuesto.

Puede encontrar una descripción de estos ataques en~\cite[cap. 4]{Hankerson:2003} y en~\cite[cap. 4]{Washington:2008}.
