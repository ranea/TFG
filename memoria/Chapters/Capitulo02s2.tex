\clearpage

\section{Curvas elípticas sobre cuerpos finitos}
\label{sec:Curvas elípticas sobre cuerpos finitos}

Sea $\Fq$ un cuerpo finito de $q$ elementos, donde $q$ es la potencia $n$-ésima de un primo $p$ y sea $E$ una curva elíptica definida sobre $\Fq$. Como el número de pares $(x, y)$ con $x, y \in \Fq$ es finito, el grupo de puntos $E(\Fq)$ es finito. En este aparatado estudiaremos las propiedades de este grupo, como el orden, que serán importantes en muchos contextos.

% TODO: ver como dejar
% Dos principales restricciones de los grupos $E(\Fq)$ se dan en los siguientes dos teoremas.
Los dos resultados más importantes se dan en los siguientes dos teoremas.

\begin{teorema}\label{th:estructura grupo puntos}
	Sea $E$ una curva elíptica definida sobre un cuerpo finito $\Fq$. Entonces
	$$
		E(\Fq) \simeq \mathbb{Z}_n \textrm{ or } \mathbb{Z}_{n_1} \oplus \mathbb{Z}_{n_2}
	$$
	para algún entero $n \ge 1$ o para algunos enteros $n_1, n_2 \ge 1$ con $n_1 \mid n_2$.
\end{teorema}
\begin{proof}
Como $E(\Fq)$ es un grupo abeliano finito, será isomorfo a una suma directa de grupos cíclicos
$$
	\mathbb{Z}_{n_1} \oplus \mathbb{Z}_{n_2} \oplus \dots \mathbb{Z}_{n_r}
$$
con $n_i \mid n_{i+1}$. Como, para cada $i$, el grupo $Z_{n_i}$ tiene $n_1$ elementos de orden un divisor de $n_1$, tenemos que $E(\Fq)$ tiene $n_1^r$ elementos de orden un divisor de $n_1$. Por el teorema~\ref{th:estructura subgrupos torsión}, hay hasta $n_1^2$ (incluso si permitimos coordenadas en la clasura algebraica de $\Fq$). Por tanto, $r \le 2$.
\end{proof}

\begin{teorema}[Teorema de Hasse]\label{th:teorema de Hasse}
	Sea $E$ una curva elíptica definida sobre un cuerpo finito $\Fq$. Entonces el orden de $E(\Fq)$ verifica
	$$
		|q + 1 - \left\vert{E(\F_{q})}\right\vert| \le 2 \sqrt{q}.
	$$
\end{teorema}
\begin{proof}
	Véase apartado~\ref{sub:Teorema de Hasse}.
\end{proof}


\subsection{Ley de grupo para cuerpos base de característica 2}
\label{sub:Ley de grupo para cuerpos base de característica 2}

En el apartado~\ref{sub:Ecuaciones de Weierstrass simplificadas}, clasificamos las curvas elípticas sobre $\Fm$ según eran supersingulares o no. Ahora podemos definir este concepto.

\begin{definicion}\label{def:supersingular}
	Sea $E$ una curva elíptica sobre un cuerpo base de característica $p$. Diremos que $E$ es \emph{supersingular} si $E[p] = \{\infty\}$.
\end{definicion}

Una ventaja de las curvas supersingulares es que los cálculos involucrados en la multiplicación de un punto por un escalar se pueden hacer con aritmética de cuerpos finitos, que es en general más rápida que la aritmética de curvas elípticas~\cite[cap. 4]{Washington:2008}. Sin embargo, estas curvas no son seguras en criptografía ya que son vulnerables a un tipo de ataques conocidos como ataques de emparejamiento Weil y Tate ~\cite[cap. 4]{Hankerson:2003}. Por este último motivo, sólo vamos a dar las fórmulas de adicción para curvas no supersingulares sobre $\Fm$.

% TODO: ver que hacer.
%En el apartado~\ref{sub:Ley de grupo} vimos las formulas de adicción cuando la característica del cuerpo base era distinta de 2 y 3. Como los cuerpos finitos de característica 2 son interesante a nivel computacional, presentamos aquí las fórmulas de adicción para las curvas no supersingulares (véase ~\ref{lm:ecuaciones curva eliptica caracteristica dos}), ya que no es recomendable usar curvas supersingulares en criptografía (véase ~\ref{}).

\begin{formulasadiccion}\label{fa:cuerpos base caracteristica 2}
Sea $E$ una curva elíptica definida por la ecuación $y^2 + x y = x^3 + a x^2 + b$ sobre el cuerpo finito $\Fm$. Las fórmulas de adicción son las siguientes:
	\begin{enumerate}[label=\alph*)]
	   \item $P + \infty = \infty + P = P,$ para todo $P \in E(\Fm)$
	   \item Si $P = (x, y) \in E(\Fm)$, entonces $(x, y) + (x, x + y) = \infty$ ya que $-P = (x, x + y)$. Además, $- \infty = \infty$.
	   \item Sea $P = (x_1, y_1) \in E(\Fm)$ y $Q = (x_2, y_2) \in E(\Fm)$, donde $P \neq \pm Q$. Entonces $P + Q = (x_3, y_3)$, donde
	   $$
	   x_3 = \lambda^2 \lambda + x_1 + x_2 + a, \quad
	   y_3 = \lambda (x_1 + x_3) + x_3 + y_1
	   $$
	   con $\lambda = (y_1 + y_2)/(x_1 + x_2)$.
	   \item Sea $P = (x_1, y_1) \in E(\Fm)$, donde $P \neq -P$. Entonces $2 P = (x_3, y_3)$ donde:
	   $$
	   x_3 = \lambda^2 \lambda + a = x_1^2 + b/x_1^2, \quad
	   y_3 = x_1^2 + \lambda x_3 + x_3
	   $$
	   con $\lambda = x_1 + y_1/x_1$.
	\end{enumerate}
\end{formulasadiccion}

% TODO: rellenar ref y ver que hacer
% Vimos en el apartado~\ref{sub:Ecuaciones de Weierstrass simplificadas} que la ecuación de una curva elíptica definida sobre $\Fm$ podía simplificarse de dos formas, dado curvas supersingulares y no supersingulares (véase ~\ref{}). Puesto que las curvas supersingulares no se utilizan en criptografía (véase ~\ref{}), trabajaremos sobre las no supersingulares. Las fórmulas de adicción para este tipo de curvas, cuya ecuación simplificada era  $y^2 + xy = x^3 + a x^2 + b$, son las siguientes:
%
% Análogamente a~\ref{th:grupo abeliano}, se puede probar que el conjunto $E(\Fm)$ junto con la operación anterior definida es un grupo abeliano.
%
% \begin{teorema}
% 	La suma~\ref{def:ley de grupo Fm} de puntos en una curva elíptica $E$ sobre el cuerpo $\Fm$ satisface la siguientes propiedades:
% 	\begin{itemize}
% 		\item \emph{Conmutatividad}: $$P_1 + P_2 = P_2 + P_1,\ \forall P_1, P_2 \in E(\Fm).$$
% 		\item \emph{Existencia de elemento neutro}: $$P + \infty = P,\ \forall P \in E(\Fm).$$
% 		\item \emph{Existencia de elemento opuesto}: $$P + (-P) = \infty,\ \forall P \in E(\Fm).$$
% 		\item \emph{Asociatividad}: $$(P_1 + P_2) + P_3 = P_1 + (P_2 + P_3), \ \forall P_1, P_2, P_3 \in E(\Fm).$$
% 	\end{itemize}
% 	En otras palabras, $(E(\Fm), +, \infty)$ es un grupo abeliano.
% \end{teorema}


\subsection{Endomorfismo de Frobenius}
\label{sub:Endomorfismo de Frobenius}

En este apartado veremos un ejemplo importante de endomorfismo. Jugará una papel crucial en la teoría de curvas elípticas sobre cuerpos finitos.

\begin{proposicion}
	Sea $E$ una curva elíptica definida sobre un cuerpo finito $\Fq$. La aplicación definida por
	$$
		\phi_q(x, y) = (x^q, y^q), \quad \phi_q(\infty) = \infty
	$$
	es un endomorfismo y se llama \emph{endormofirsmo de Frobenius}.
\end{proposicion}
\begin{proof}
Veamos primero que $\phiq(x, y) \in \EFqca$. Usaremos las siguientes propiedades de los cuerpos finitos:
\begin{align*}
	(a + b)^q &= a^q + b^q, \ \forall a, b \in \Fqca \\
	a^q &= a, \ \forall a \in \Fq.
\end{align*}
Como la demostración es esencialmente la misma para la ecuación de Weirstrass general y la ecuación simplificada para cuerpos base de característica distinta de 2 y 3, usaremos está última. Tenemos
$$
	y^2 = x^3 + a x + b,
$$
donde $a, b \in Fq$. Elevando todo a la potencia $q$-ésima obtenemos
$$
	(y^q)^2 = (x^q)^3 + a (x^q) + b.
$$
Luego $(x^q, y^q)$ está en $E$. Veamos ahora $\phiq$ es un homorfismo. Sea $(x_1, y_1), (x_2, y_2) \in \EFqca$ con $x_1 \neq x_2$. La suma es $(x_3, y_3)$ con
$$
	x_3 = m^2 - x_1 - x_2, \ y_3 = m (x_1 - x_3) - y_1, \textrm{ donde } m = \frac{y_2 - y_1}{x_2 - x_1}
$$
Elevando todo a la potencia $q$-ésima obtenemos
$$
	x_3^q = m'^2 - x_1^q - x_2^q, \ y_3^q = m' (x_1^q - x_3^q) - y_1^q, \textrm{ donde } m' = \frac{y_2^q - y_1^q}{x_2^q - x_1^q}.
$$
Por tanto,
$$
	\phiq(x_3, y_3) = \phiq(x_1, y_1) + \phiq(x_2, y_2).
$$
Los casos donde $x_1 = x_2$ o uno de los dos puntos es $\infty$ se comprueban fácilmente. Sin embargo, el caso de añadir un punto consigo mismo presenta una sutileza. Usando las fórmulas, tenemos que $2 (x_1, y_1) = (x_3, y_3)$ donde
$$
	x_3 = m^2 - 2 x_1, \ y_3 = m (x_1 - x_3) - y_1, \textrm{ donde } m = \frac{3 x_1^2 + a}{2 y_1}.
$$
Cuando elevamos a la potencia $q$-ésima, obtenemos
$$
	x_3^q = m'^2 - 2^q x_1^q, \ y_3^q = m' (x_1^q - x_3^q) - y_1^q, \textrm{ donde } m' = \frac{3^q (x_1^q)^2 + a^q}{2^q y_1^q}.
$$
Como $2, 3, a \in \Fq$, tenemos $2^q = 2,\ 3^q = 3,\ a^q = a$. Luego hemos obtenido la fórmula para duplicar el punto $(x_1^q, y_1^q)$ de $E$.

Finalmente, como $\phiq$ es un homomorfismo dado por funciones racionales, es un endomorfismo de $E$.
\end{proof}

% TODO: ver que hacer
% En general, para una curva eliptica arbitraria R definida sobre Fq y una extension F de Fq, el endomorfismo de Frobenius phiq permuta los elementos de E(F) y es la identidad en el subgrupo E(Fq).
Nótese que este endomorfismo no es más que aplicar el \emph{automorfismo de Frobenius}
\begin{align*}
	\Fqca &\to \Fqca \\
	x &\mapsto x^q
\end{align*}
sobre las componentes de un punto. Veamos ahora una cuantas propiedades de este endomorfismo.

\begin{lema}\label{lm:elementos fijos endomorfismo Frobenius}
	Sea $E$ una curva elíptica definida sobre un cuerpo finito $\Fq$ y sea $(x, y) \in \EFqca$. Entonces
	$$
		(x, y) \in E(\Fq) \iff \phiq(x, y) = (x, y).
	$$
\end{lema}
\begin{proof}
Usando $a^q = a, \ \forall a \in \Fq$, tenemos
	\begin{align*}
		(x, y) \in E(\Fq) &\iff x, y \in \Fq \\
			&\iff \phiq(x) = x \textrm{ y } \phiq(y) = y \\
			&\iff \phiq(x, y) = (x, y).
	\end{align*}
\end{proof}

\begin{lema}
	Sea $E$ una curva elíptica definida sobre un cuerpo finito $\Fq$. Entonces $\phi_q$ no es separable y es de grado $q$.
\end{lema}
\begin{proof}

Como $q = 0$ en $\Fq$, la derivada de $x^q$ es idénticamente cero, luego $\phiq$ no es separable. Por otro lado, $\phiq$ tiene grado $q$ aplicando la definición de grado de un endomorfismo.
\end{proof}

% TODO: rebajar a lema?
\begin{proposicion}\label{pp:separabilidad endomorfismo frobenius menos uno}
	Sea $E$ una curva elíptica definida sobre un cuerpo finito $\Fq$, donde $q$ es una potencia de un primo $p$. Sean $r, s$ enteros no nulos. El endomorfismo $r \phi_q + s$ es separable si y solo si $p \nmid s$.
\end{proposicion}
\begin{proof}
Podemos escribir el endomorfismo multiplicación por $r$ como
$$
	r(x, y) = (R_r(x), y S_r(x))
$$
por el lema~\ref{lm:endomorfismo con funciones racionales}. Entonces
\begin{align*}
	(R_{r \phiq}(x), y S_{r \phiq}(x)) &= (\phiq r) ( x, y) = (R_r^q(x), y^q S_r^q(x)) \\
	&= (R_r^q(x), y(x^3 + a x + b)^{(q-1)/2} S_r^q(x)).
\end{align*}
Así,
$$
	c_{r \phiq} = R_{r \phiq}' / S_{r \phiq} = q R_r^{q - 1} R_r' / S_{r \phiq} = 0.
$$
Además, $c_s = R_s' / S_s = s$ por la proposición~\ref{pp:endomorfismo multiplicación}. Por el lema~\ref{lm:cociente derivadas funciones racionales},
$$
	R_{r \phiq + s}' / S_{r \phiq + s} = c_{r \phiq + s} = c_{r \phiq} + c_{s} = 0 + s = s
$$
Por lo tanto, $r \phi_q + s$ es separable si y solo si $R_{r \phiq + s}' \neq 0$ y esto ocurre si y solo si $p \nmid s$.
\end{proof}

\begin{proposicion}\label{pp:relacion nucleo y endomorfismo Frobenius}
	Sea $E$ una curva elíptica definida sobre un cuerpo finito $\Fq$ y consideremos el endomorfismo $\phiq^n - 1$ con $n \ge 1$. Entonces
	\begin{enumerate}
		\item $\ker(\phiq^n - 1) = E(\F_{q^n})$.
		\item $\phiq^n - 1$ es separable, por lo que $\left\vert{E(\F_{q^n})}\right\vert = \deg(\phiq^n - 1)$.
	\end{enumerate}
\end{proposicion}
\begin{proof}
Como $\phiq^n$ es el endomorfismo de Frobenius para el cuerpo $\F_{q^n}$, (1) es consecuencia del lema~\ref{lm:elementos fijos endomorfismo Frobenius}. (2) se tiene aplicando las proposiciones~\ref{pp:separabilidad endomorfismo frobenius menos uno} y~\ref{pp:cardinal del núcleo}.
\end{proof}

\begin{lema}\label{lm:descomposicion grado frobenius}
	Sean $r, s$ enteros. Entonces $\deg(r \phi_q - s) = r^2 q + s^2 - rsa$.
\end{lema}
\begin{proof}
La proposición~\ref{pp:grado endomorfismo a*alpha + b*beta} implica que
\begin{align*}
	\deg(r \phi_q - s) =& \ r^2 \deg(\phiq) + s^2 \deg(-1) \\
	&+ r s (\deg(\phiq - 1) - \deg(\phiq) - \deg(-1)).
\end{align*}
Como $\deg(\phiq) = q$ y $\deg(-1) = 1$, el resultado se deduce de~\ref{pp:relacion nucleo y endomorfismo Frobenius} y~\ref{th:estructura grupo puntos}.
\end{proof}

\subsection{Teorema de Hasse}
\label{sub:Teorema de Hasse}

Tenemos ya todas las herramientas para probar el teorema de Hasse.

\begin{teorema2}[\ref{th:teorema de Hasse}]
	Sea $E$ una curva elíptica definida sobre un cuerpo finito $\Fq$. Entonces el orden de $E(\Fq)$ verifica
	$$
		|q + 1 - \left\vert{E(\F_{q})}\right\vert| \le 2 \sqrt{q}.
	$$
\end{teorema2}
\begin{proof}[Demostración del teorema~\ref{th:teorema de Hasse}]
Sea
$$
	a = q + 1 - \left\vert{E(\F_{q})}\right\vert = q + 1 - deg(\phiq - 1)
$$
donde estamos usando proposición~\ref{pp:relacion nucleo y endomorfismo Frobenius}. Veamos que $|a| \le 2 \sqrt{q}$.

Como $\deg(r \phi_q - s) \ge 0$, el lema~\ref{lm:descomposicion grado frobenius} implica que
$$
q \left( \frac{r}{s} \right)^2 - a \left(\frac{r}{s} \right) ^2 + 1 \ge 0.
$$
para cualesquiera enteros $r, s$. Como el conjunto de los racionales es \emph{denso} en $\mathbb{R}$, tenemos que
$$
q x^2 - a x + 1 \ge 0
$$
para todo número real $x$. Así, el discriminante de este polinomio es negativo o cero, lo que implica que $a^2 - 4 q \le 0$, luego $| a | \le 2 \sqrt{q}$.
\end{proof}

\begin{nota}[comentarios del teorema~\ref{th:teorema de Hasse}]\leavevmode
	\begin{itemize}
		\item Como la ecuación de Weierstrass tiene al menos dos soluciones para todo $x \in \Fq$, una primera acotación del orden de $E(\Fq)$ es  $\left\vert{E(\F_{q})}\right\vert \in [1, 2 q + 1]$. El teorema de Hasse proporciona cotas más optimas:
		$$
		\left\vert{E(\F_{q})}\right\vert \in [q + 1 - 2 \sqrt{q}, \ q + 1 + 2 \sqrt{q}].
		$$
		Como $2\sqrt{q}$ es pequeño respecto a $q$, $E(\Fq) \approx q$.
		\item Tres resultados fueron clave para la demostración:
			\begin{itemize}
				\item La identificación de $E(\Fq)$ con el núcleo de $\phiq - 1$.
				\item La igualdad entre el orden del núcleo y el grado de $\phiq - 1$ gracias a la separabilidad de $\phiq - 1$
				\item El emparejamiento Weil, especialmente la parte (6) del teorema~\ref{pp:endomorfismo Weil} y su consecuencia~\ref{pp:relacion nucleo y endomorfismo Frobenius}.
			\end{itemize}
	\end{itemize}
\end{nota}

% TODO: probar caracterizacion supersingularidad?
