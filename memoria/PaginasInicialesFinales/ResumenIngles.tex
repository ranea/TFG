%*******************************************************
% Abstract
%*******************************************************
%\renewcommand{\abstractname}{Abstract}
\pdfbookmark[1]{Abstract}{Abstract}
\begingroup
\let\clearpage\relax
\let\cleardoublepage\relax
\let\cleardoublepage\relax

\begin{otherlanguage}{american}

\chapter*{Abstract}

% Deberá estar escrito completamente en inglés y tener una longitud mínima de 1500 palabras. Igualmente aparecerán las palabras clave en inglés.

In this project, it is studied elliptic curve theory, its application in cryptography and elliptic curve cryptographic protocols. It is also explained the computer programm we have developed in this field.

Every main topic is going to be explained briefly starting with elliptic curve theory.

\section*{Elliptic curve theory}

First, an elliptic curve is defined and some concepts related to such as rational points, discriminant, point at infinity or base field. Some examples of elliptic curves over the field of real numbers and their graphs are given.

Then, the general Weierstrass equation shown in the elliptic curve definition is simplified by changes of variable depends on the base field characteristic. If the characteristic is different from two or three, the general equation can be replaced with the short Weierstrass equation. The simplified equation for elliptic curves with base field of characteristic two is given as well.

After that, the chord-and-tangent method is explained. This is a geometric method, given two points on an elliptic curve, to produce a third point on the curve. The basic idea of this method is drawing the line between the two points, taking the point on the elliptic curve which intersect this line and reflecting this last point over the x-axis. If the two points given are the same, the method changes slightly.

An addition law for the set of points of an elliptic curve with explicit algebraic formulas are given by inspiring in the chord-and-tangent method. Only the formulas for the elliptic curves defined over a field with characteristic different from two or three are given. Later, we will show the formulas for elliptic curve defined over the finite field with characteristic two.

After proving these formulas define a valid binary operation, we prove the first theorem. This result affirms that the set of points of an elliptic curve with this composition law forms an abelian group. The fact that the set of rational points with this law is a group, is a first step to apply elliptic curves in cryptography.

The next step is the double-and-add method. This algorithm allows to compute the addition of a point to itself much more efficient than the naïve method in some base fields such as the finite fields.

Then, projective space is introduced as a quotient space. It is considered the equivalence classes, called projective points, and it is defined some concepts such as line at infinity, points at infinity projective form of a Weierstrass equation and projective coordinates.

After this, endomorphisms of elliptic curves are studied. The concept of endomorphism is defined and a simplified representation with rational functions is given for endomorphisms of an elliptic curve defined over a base field whose characteristic is different from two or three. Then, some key concepts such as the degree or the separability of an endomorphism are defined. After these definitions, some results about endomorphisms are proved: a characterization of separability, some relations between the degree and the kernel of an endomorphism,  the surjectivity of endomorphisms, the separability of the multiplication endomorphism and a technical result about the sum of endomorphisms.

Torsion points and torsion subgroups are the next step to be studied. The previous results are used in order to prove a crucial theorem about the structure of torsion subgroups. In this proof, the multiplication endomorphism and the structure theorem for finite abelian groups play important roles.

It is possible to associate an endomorphism of an elliptic curve to a matrix with integers entries thanks to this theorem. The target is computing the degree of an endomorphism by using the determinant of the associate matrix. This simplification allows to compute the degree of a linear combination of endomorphisms which will be necessary to prove Hasse's theorem.

In order to prove the previous simplification, it is necessary the Weil pairing. This pairing between the product of torsion subgroups and the group of nth roots of unity has many useful properties. By using these properties, it is computed the degree of a linear combination of endomorphisms.

We move forward and we consider elliptic curves over finite fields. These curves are the ones which are used in public-key cryptography.

We start with a special case of the previous general theorem about the subgroup structure, but in this case it is applied to elliptic curves over finite fields.

Algebraic formulas for the addition law are given for elliptic curves over finite fields with characteristic two. The concept of supersingularity arises in this kind of elliptic curves.

Then, a well-known example of endomorphism of elliptic curves over finite field is shown, the Frobenius endomorphism. Some properties about the Frobenius endomorphism are proved. These results are about its separability, its degree or the relation between the group of rational points and the kernel of a Frobenius endomorphism variant.

By using almost all the results proved since the beginning, Hasse's theorem is proved. The most crucial results used in the proof are the identification between the group of rational points and the kernel of a Frobenius endomorphism variant, the separability of this endomorphism, the relation between the kernel and the degree of an endomorphism, and last but not least the Weil pairing which simplified the degree computation of a linear endomorphism combination.

Elliptic curve theory is left and we continue with its application in cryptography.


\section*{Elliptic curve cryptography}

To begin with, we discuss the intractable problems which some public-key schemes are based on. These are the integer factorization problem and the discrete logarithm problem. System examples based on each problem are RSA and elliptic curve systems.

An estimation of the key sizes for RSA and elliptic curve systems is shown for different security levels. This estimation demonstrates that elliptic curve systems need smaller parameters rather than RSA for the same level of security. In fact, private-key operations for elliptic curve cryptography are more efficient than for RSA, although public-key operations for RSA are faster than for ECC. Anyway, elliptic curve cryptography can be very useful in environments with power consumption, storage, operations or processing power limited thanks to its smaller parameters.

Then, the elliptic curve discrete logarithm problem is defined. The security of elliptic curve public-key cryptosystems depends on the hardness of this problem. Therefore, the most important attacks on the discrete logarithm problem are considered such as Pollard's rho algorithm, Pohling-Hellman algorithm and isomorphism attacks. For each one of the attacks presented, conditions over the discrete logarithm problem are given in order to resist these attacks.

In order to describe elliptic curve cryptographic protocol later, some elements must be described before. One of them is the domain parameters. These parameters describe an elliptic curve, the finite field which the elliptic curve is defined over, a base point and the order of the point. The other one is the key pair. It is described the generation of the public and private key, which it is based on the elliptic curve discrete logarithm problem.

Some examples of elliptic curve cryptography protocol are given. The first one is a key-agreement protocol, the Elliptic Curve Diffie-Hellman. This procedure allows two participants to securely exchange a point on an elliptic curve, although none of them initially knows the point. The second protocol is a digital signature scheme, the Elliptic Curve Digital Signature Algorithm. This procedure allows a participant to sign a digital document and another participant to verify the signature is valid.

Later, the computer developed programm is described. It has been written in python 3 and it hasn't used any external library in its logic, although some external libraries have been used to generate the documentation and to test the software.

It is split in four main modules: elemental arithmetic, finite fields, elliptic curves and cryptographic schemes. The first module implements the integers modulo a prime and polynomials with integers coefficients. The second module implements the element of a finite field and the arithmetic operations in these fields. The third module implements the group of points of some kind of elliptic curves such as elliptic curve the field of rational numbers and elliptic curves over finite field of characteristic different from two and three and of characteristic two. The last module implements the Elliptic Curve Diffie-Hellman and Elliptic Curve Digital Signature Algorithm protocols. There is an additional module which contains a list of common parameters domain used in real-world applications.

A complete programm documentation with many examples of usage has been attached. The documentation has been written in HTML with the help of the sphinx library to provide an user-friendly dynamic way to show all the functions, classes and modules and how they interact.

All the functionality has been tested, not only with unit tests but also with property-based tests. It is described the ideas behind property-based testing which is extremely powerful in application with many algebraic elements like ours. The library hypothesis has been used to write this kind of tests.

In the last chapter, conclusions are included about the project and it is discussed some ways of future development such as identity based cryptography, hyperelliptic curves or a more deep study about the attacks on the elliptic curve discrete logarithm problem.

\bigskip

\textbf{Keywords}: elliptic curve theory, public-key cryptography, elliptic curve cryptography, finite fields, cryptographic protocols.

\end{otherlanguage}

\endgroup

\vfill
