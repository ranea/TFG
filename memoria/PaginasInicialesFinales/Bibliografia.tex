%********************************************************************
% Bibliografía
%*******************************************************

% Se incluirán tanto las fuentes primarias como todas aquellas cuyo peso haya sido menor en la realización del trabajo. Se recomienda un breve comentario de las referencias, ya sea individualizado, por grupos de referencias o global. En caso de incluir URLs de páginas web deberán ir acompañadas de título, autor y fecha de último acceso, entre otros datos relevantes. Se recomienda no abusar de este tipo de fuentes.


% work-around to have small caps also here in the headline
\manualmark
\markboth{\spacedlowsmallcaps{\bibname}}{\spacedlowsmallcaps{\bibname}} % work-around to have small caps also
%\phantomsection
\refstepcounter{dummy}
\addtocontents{toc}{\protect\vspace{\beforebibskip}} % to have the bib a bit from the rest in the toc
\addcontentsline{toc}{chapter}{\tocEntry{\bibname}}
\label{app:bibliography}

\printbibliography

% TODO: escribir comentario de bibliografía

\cleardoublepage

\chapter*{Referencias web}

Documentación de las herramientas utilizadas en el desarrollo del programa:

\begin{itemize}
    \item Georg Brandl. \emph{Sphinx 1.4.4 documentation}. \url{http://www.sphinx-doc.org/}, último acceso el 10 de julio de 2016.
    \item David R. MacIver. \emph{Hypothesis 3.4.1 documentation}. \url{http://hypothesis.readthedocs.io/en/latest/}, último acceso el 10 de julio de 2016.
    \item Amit Patel et al. \emph{Google Python Style Guide}. \url{http://google.github.io/styleguide/pyguide.html}, último acceso el 10 de julio de 2016.
\end{itemize}

Recopilación de parámetros de dominio de diferentes estándares utilizados en criptografía con curvas elípticas:

%\begin{enumerate}[resume]
\begin{itemize}
    \item Daniel J. Bernstein and Tanja Lange. \emph{SafeCurves: choosing safe curves for elliptic-curve cryptography}. \url{http://safecurves.cr.yp.to/}, último acceso el 10 de julio de 2016.
\end{itemize}

\label{app:bibliography end}
